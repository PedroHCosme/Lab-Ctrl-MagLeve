\section{Conclusão}

Este trabalho apresentou o processo de identificação e controle para um sistema de levitação magnética. A metodologia de identificação por resposta em frequência, baseada em \cite{kawakami2003}, permitiu a estimação dos parâmetros de um modelo linearizado de segunda ordem para a planta.

Um ponto crucial do trabalho foi a validação experimental, que revelou uma discrepância no ganho do modelo inicialmente identificado. A investigação e correção desse erro foram fundamentais e demonstraram a importância de confrontar o modelo teórico com o comportamento da planta real. Após o ajuste do modelo, foi possível projetar um controlador que atendeu com sucesso aos objetivos de estabilização e seguimento de referência.

Conclui-se que o processo foi bem-sucedido, resultando em um modelo matemático validado e um controlador funcional, além de proporcionar uma valiosa experiência prática sobre os desafios e o processo iterativo de modelagem e controle de sistemas físicos.