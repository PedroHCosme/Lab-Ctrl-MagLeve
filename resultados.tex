\section{Resultados da Identificação}

Seguindo a metodologia descrita, o sistema foi excitado com um sinal chirp, e os ganhos foram medidos para cada ponto.

\subsection{Dados Experimentais e Estimação Inicial}
Os ganhos experimentais medidos estão consolidados na Tabela \ref{tab:ganhos}.

% --- INSIRA SEUS DADOS DE GANHO MEDIDOS AQUI ---
\begin{table}[h]
    \centering
    \caption{Ganhos medidos experimentalmente.}
    \label{tab:ganhos}
    \begin{tabular}{cc}
        \toprule
        \textbf{Frequência (Hz)} & \textbf{Ganho (|Y/U|)} \\
        \midrule
        X & Y \\ % Exemplo
        X & Y \\ % Exemplo
        % ... adicione as outras medições aqui ...
        X & Y \\ % Exemplo
        \bottomrule
    \end{tabular}
\end{table}

A partir desses dados, os parâmetros da função de transferência (\ref{eq:ft_planta}) foram estimados via mínimos quadrados. Os valores obtidos para a estimação inicial estão na Tabela \ref{tab:parametros_inicial}.

% --- INSIRA SEUS PARÂMETROS CALCULADOS (COM O ERRO) AQUI ---
\begin{table}[h]
    \centering
    \caption{Parâmetros do modelo linearizado (estimação inicial).}
    \label{tab:parametros_inicial}
    \begin{tabular}{cc}
        \toprule
        \textbf{Parâmetro} & \textbf{Valor Estimado} \\
        \midrule
        $\hat{\eta}$ & XX.X \\ % Substitua pelo seu valor de eta
        $\hat{A}$    & YY.Y \\ % Substitua pelo seu valor de A
        \bottomrule
    \end{tabular}
\end{table}

