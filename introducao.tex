\section{Introdução e Objetivo}

Sistemas de levitação magnética têm despertado crescente interesse devido às suas aplicações inovadoras em áreas como transporte de alta velocidade e rolamentos industriais, onde a ausência de contato físico resulta em menor atrito e maior eficiência energética \cite{kawakami2003,feedback2006control}. Do ponto de vista da engenharia de controle, o levitador magnético do tipo atrativo é um sistema particularmente desafiador por ser inerentemente instável em malha aberta e possuir uma dinâmica não linear \cite{kawakami2003}.

A força gerada pelo eletroímã é proporcional ao quadrado da corrente aplicada e inversamente proporcional ao quadrado da distância da esfera. Portanto, a obtenção de um modelo matemático preciso que descreva o comportamento do sistema em torno de um ponto de operação é um passo fundamental para o projeto de controladores eficazes.

Diante disso, este trabalho tem como objetivo principal a identificação dos parâmetros de um modelo matemático linear para a planta de levitação magnética (Maglev) da Feedback Instruments. A metodologia empregada baseia-se na análise da resposta em frequência do sistema em malha fechada, conforme proposto por Kawakami et al. \cite{kawakami2003}, permitindo uma identificação segura e precisa da dinâmica da planta a partir de dados experimentais.