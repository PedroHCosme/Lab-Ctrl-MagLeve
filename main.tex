\documentclass[conference,harvard,brazil,english]{sbatex}

% --- Codificação e Fontes ---
\usepackage[utf8]{inputenc}
\usepackage[T1]{fontenc}
\usepackage{lmodern}

% --- Pacotes Recomendados ---
\usepackage{graphicx}
\usepackage{amsmath}
\usepackage{amssymb}
\usepackage{booktabs}
\usepackage{microtype}
\usepackage[hyphens]{url}
\usepackage{hyperref}

\makeatletter
\def\verbatim@font{\normalfont\ttfamily\footnotesize}
\makeatother

\begin{document}

\title{Módulo 1 - Processo Massa Mola}


\author{Pedro Henrique de Menezes Cosme}{pedrocosme@ufmg.br}
\author[1]{Pedro Henrique Silva Lima}{pedroh404@ufmg.br}


\address{Universidade Federal de Minas Gerais (UFMG)}


\twocolumn[
\maketitle


\selectlanguage{english}
\begin{abstract}
This report details the process of identifying the parameters of a linearized model for a Feedback Instruments magnetic levitation (Maglev) system. The procedure is based on a frequency-domain identification technique, where the system is excited with sinusoidal signals in a closed-loop configuration. The parameters of the second-order unstable model were estimated using the least-squares method, and the resulting model was validated by comparing its frequency response with experimental data.
\end{abstract}

\keywords{System Identification, Magnetic Levitation, Maglev, Frequency Response.}

\selectlanguage{brazil}
\begin{abstract}
Este relatório detalha o processo de identificação dos parâmetros de um modelo linearizado para a planta de levitação magnética (Maglev) da Feedback Instruments. O procedimento baseia-se em uma técnica de identificação no domínio da frequência, na qual o sistema é excitado com sinais senoidais em uma configuração de malha fechada. Os parâmetros do modelo instável de segunda ordem foram estimados pelo método dos mínimos quadrados, e o modelo resultante foi validado pela comparação de sua resposta em frequência com os dados experimentais.
\end{abstract}

\keywords{Identificação de Sistemas, Levitação Magnética, Maglev, Resposta em Frequência.}
]

\selectlanguage{brazil}

% --- INCLUSÃO DAS SEÇÕES DO RELATÓRIO ---
\section{Introdução e Objetivo}

Sistemas de levitação magnética têm despertado crescente interesse devido às suas aplicações inovadoras em áreas como transporte de alta velocidade e rolamentos industriais, onde a ausência de contato físico resulta em menor atrito e maior eficiência energética \cite{kawakami2003,feedback2006control}. Do ponto de vista da engenharia de controle, o levitador magnético do tipo atrativo é um sistema particularmente desafiador por ser inerentemente instável em malha aberta e possuir uma dinâmica não linear \cite{kawakami2003}.

A força gerada pelo eletroímã é proporcional ao quadrado da corrente aplicada e inversamente proporcional ao quadrado da distância da esfera. Portanto, a obtenção de um modelo matemático preciso que descreva o comportamento do sistema em torno de um ponto de operação é um passo fundamental para o projeto de controladores eficazes.

Diante disso, este trabalho tem como objetivo principal a identificação dos parâmetros de um modelo matemático linear para a planta de levitação magnética (Maglev) da Feedback Instruments. A metodologia empregada baseia-se na análise da resposta em frequência do sistema em malha fechada, conforme proposto por Kawakami et al. \cite{kawakami2003}, permitindo uma identificação segura e precisa da dinâmica da planta a partir de dados experimentais.
\section{Descrição da Planta}

A planta utilizada neste trabalho é o sistema de Levitação Magnética (Maglev) 33-210 da Feedback Instruments \cite{feedback2006install}. Seus principais subsistemas são:

\begin{itemize}
    \item \textbf{Unidade Mecânica:} Composta por uma base com interface de conexão, um eletroímã (bobina) montado na parte superior, uma esfera metálica e um par de sensores infravermelhos (emissor e receptor) que medem a posição vertical da esfera.
    \item \textbf{Sistema de Controle:} O controle em tempo real é realizado por um computador equipado com uma placa de aquisição de dados (Advantech PCI1711), que se comunica com a planta através de uma interface analógica (33-301) \cite{feedback2006install}. Os algoritmos de controle são implementados no ambiente MATLAB/Simulink. O sistema envia sinais de tensão para um driver de corrente, que alimenta a bobina para gerar o campo magnético.
\end{itemize}

\subsection{Variáveis do Processo}
As principais variáveis envolvidas no processo de controle são:
\begin{itemize}
    \item \textbf{Variável Manipulada:} Tensão de controle ($u$) aplicada ao driver de corrente. Esta tensão é convertida em uma corrente ($i$) que flui pela bobina do eletroímã.
    \item \textbf{Variável Medida:} Posição vertical da esfera ($H$), medida pelo sensor infravermelho e convertida em um sinal de tensão ($Y$).
    \item \textbf{Variável de Referência:} Posição desejada para a esfera, ou setpoint ($Y_{ref}$).
\end{itemize}

\subsection{Não-Linearidades}
A principal característica não linear do sistema é a força de atração eletromagnética ($F_M$), que é descrita pela seguinte equação \cite{kawakami2003,feedback2006control}:
\begin{equation}
    F_{M} = k \frac{I^{2}}{H^{2}}
    \label{eq:forca_magnetica}
\end{equation}
onde $k$ é uma constante eletromecânica, $I$ é a corrente na bobina e $H$ é a distância entre a esfera e o eletroímã. Essa relação quadrática exige que o sistema seja linearizado para a aplicação de técnicas clássicas de projeto de controladores.
\section{Modelagem Matemática e Identificação}

A dinâmica vertical da esfera suspensa pode ser descrita pela Segunda Lei de Newton, resultando na seguinte equação diferencial não linear \cite{kawakami2003}:

\begin{equation}
    m\frac{d^{2}H}{dt^{2}} = mg - k\frac{I^{2}}{H^{2}}
\end{equation}

onde $m$ é a massa da esfera e $g$ é a aceleração da gravidade.

\subsection{Linearização do Modelo}
Para o projeto de controladores lineares, o modelo é linearizado através de uma expansão em série de Taylor de primeira ordem em torno de um ponto de equilíbrio ($H_0$, $I_0$). Considerando pequenas variações em torno deste ponto ($h = H - H_0$ e $i = I - I_0$), a equação dinâmica linearizada se torna:

\begin{equation}
    m\frac{d^{2}h}{dt^{2}} = \lambda h - \beta i
\end{equation}
onde $\lambda = \frac{2mg}{H_0}$ e $\beta = \frac{2kI_0}{H_0^2}$.

Considerando as relações lineares do sensor e do atuador ($y = \gamma h$ e $i = \rho u$, onde $y$ e $u$ são as variações da tensão de saída e de controle), a dinâmica pode ser expressa em termos de variáveis de tensão. Aplicando a Transformada de Laplace, obtemos a função de transferência do sistema \cite{kawakami2003}:

\begin{equation}
    G(s) = \frac{Y(s)}{U(s)} = -\frac{A}{s^{2} - \eta}
    \label{eq:ft_planta}
\end{equation}

onde $A$ e $\eta$ são constantes positivas. A presença de um polo no semiplano direito ($s = +\sqrt{\eta}$) confirma a instabilidade do sistema em malha aberta. O objetivo da identificação é estimar os valores de $A$ e $\eta$ experimentalmente.

\subsection{Método de Identificação por Resposta em Frequência}
O método de identificação adotado consiste em analisar a resposta do sistema em malha fechada a uma perturbação senoidal \cite{kawakami2003}. O sistema é primeiramente estabilizado por um controlador analógico. Em seguida, um sinal senoidal de pequena amplitude é somado à entrada do sistema.

Para uma entrada senoidal, a resposta em frequência teórica do modelo linearizado (\ref{eq:ft_planta}) é puramente real:
\begin{equation}
    G(j\omega) = \frac{A}{\omega^{2} + \eta}
\end{equation}

O ganho do sistema em uma dada frequência $\omega$ é, portanto, $|G(j\omega)|$. Medindo-se o ganho experimentalmente em $N$ frequências distintas ($\omega_1, \omega_2, ..., \omega_N$), obtemos um sistema de equações lineares:
\begin{equation}
    A - \eta \cdot \text{Gain}(\omega_i) = \omega_i^2 \cdot \text{Gain}(\omega_i), \quad i=1,...,N
\end{equation}

Este sistema pode ser escrito na forma matricial $P\theta = Q$, onde $\theta = [\eta, A]^T$ é o vetor de parâmetros a ser estimado. A solução que minimiza o erro quadrático é dada por \cite{kawakami2003}:
\begin{equation}
    \hat{\theta} = (P^T P)^{-1} P^T Q
\end{equation}
\section{Resultados da Identificação}

Seguindo a metodologia descrita, o sistema foi excitado com um sinal chirp, e os ganhos foram medidos para cada ponto.

\subsection{Dados Experimentais e Estimação Inicial}
Os ganhos experimentais medidos estão consolidados na Tabela \ref{tab:ganhos}.

% --- INSIRA SEUS DADOS DE GANHO MEDIDOS AQUI ---
\begin{table}[h]
    \centering
    \caption{Ganhos medidos experimentalmente.}
    \label{tab:ganhos}
    \begin{tabular}{cc}
        \toprule
        \textbf{Frequência (Hz)} & \textbf{Ganho (|Y/U|)} \\
        \midrule
        X & Y \\ % Exemplo
        X & Y \\ % Exemplo
        % ... adicione as outras medições aqui ...
        X & Y \\ % Exemplo
        \bottomrule
    \end{tabular}
\end{table}

A partir desses dados, os parâmetros da função de transferência (\ref{eq:ft_planta}) foram estimados via mínimos quadrados. Os valores obtidos para a estimação inicial estão na Tabela \ref{tab:parametros_inicial}.

% --- INSIRA SEUS PARÂMETROS CALCULADOS (COM O ERRO) AQUI ---
\begin{table}[h]
    \centering
    \caption{Parâmetros do modelo linearizado (estimação inicial).}
    \label{tab:parametros_inicial}
    \begin{tabular}{cc}
        \toprule
        \textbf{Parâmetro} & \textbf{Valor Estimado} \\
        \midrule
        $\hat{\eta}$ & XX.X \\ % Substitua pelo seu valor de eta
        $\hat{A}$    & YY.Y \\ % Substitua pelo seu valor de A
        \bottomrule
    \end{tabular}
\end{table}


\section{Projeto do Controlador e Validação}

Com base no modelo inicialmente identificado, foi projetado um controlador com o objetivo de estabilizar a planta e garantir o seguimento de referência.

\subsection{Falha na Implementação e Diagnóstico do Erro}
O primeiro controlador foi projetado utilizando o modelo com os parâmetros da Tabela \ref{tab:parametros}. No entanto, ao ser implementado na planta real, o controlador não funcionou como o esperado. A ação de controle mostrou-se desproporcional, resultando em um comportamento instável e saturação do atuador.

Após análise, com o auxílio do professor, foi diagnosticado que o modelo matemático, embora representasse bem a dinâmica (polos), continha um erro significativo no ganho estático. Verificou-se que o ganho estimado ($\hat{A}_{inicial}$) estava incorreto por um fator de aproximadamente []. Esse erro de propagação, possivelmente ocorrido durante a conversão de unidades ou coleta de dados, foi a causa da falha do controlador inicial.

\subsection{Correção do Modelo e Novo Projeto}
Para corrigir o modelo, foi aplicado um fator de correção ao ganho $\hat{A}$. O novo valor ajustado do parâmetro é apresentado na Tabela \ref{tab:parametros_corrigido}.

% --- INSIRA O PARÂMETRO A CORRIGIDO AQUI ---
\begin{table}[h]
    \centering
    \caption{Parâmetro de ganho do modelo corrigido.}
    \label{tab:parametros_corrigido}
    \begin{tabular}{cc}
        \toprule
        \textbf{Parâmetro} & \textbf{Valor Corrigido} \\
        \midrule
        $\hat{A}_{corrigido}$ & % Insira o valor de A corrigido aqui
        \\
        \bottomrule
    \end{tabular}
\end{table}

De posse do modelo corrigido, um novo controlador foi projetado. A estratégia utilizada foi []. Os ganhos do controlador final são [].

\subsection{Resultados do Controlador Ajustado}
O controlador projetado com o modelo corrigido foi implementado na planta e, desta vez, apresentou um desempenho satisfatório. As Figuras \ref{fig:resp_degrau} e \ref{fig:resp_senoide} mostram a resposta do sistema em malha fechada para uma referência de onda quadrada e senoidal, respectivamente.

% --- INSIRA SEUS GRÁFICOS DE RESULTADOS AQUI ---
\begin{figure}[h]
    \centering
    % \includegraphics[width=\columnwidth]{grafico_degrau.png}
    \caption{Resposta do sistema controlado a uma referência de onda quadrada.}
    \label{fig:resp_degrau}
\end{figure}

\begin{figure}[h]
    \centering
    % \includegraphics[width=\columnwidth]{grafico_senoide.png}
    \caption{Resposta do sistema controlado a uma referência senoidal.}
    \label{fig:resp_senoide}
\end{figure}

Os resultados demonstram que o sistema foi capaz de seguir as referências com estabilidade e precisão, validando tanto o modelo corrigido quanto o controlador projetado.
\section{Conclusão}

Este trabalho apresentou o processo de identificação e controle para um sistema de levitação magnética. A metodologia de identificação por resposta em frequência, baseada em \cite{kawakami2003}, permitiu a estimação dos parâmetros de um modelo linearizado de segunda ordem para a planta.

Um ponto crucial do trabalho foi a validação experimental, que revelou uma discrepância no ganho do modelo inicialmente identificado. A investigação e correção desse erro foram fundamentais e demonstraram a importância de confrontar o modelo teórico com o comportamento da planta real. Após o ajuste do modelo, foi possível projetar um controlador que atendeu com sucesso aos objetivos de estabilização e seguimento de referência.

Conclui-se que o processo foi bem-sucedido, resultando em um modelo matemático validado e um controlador funcional, além de proporcionar uma valiosa experiência prática sobre os desafios e o processo iterativo de modelagem e controle de sistemas físicos.

% --- BIBLIOGRAFIA ---
\nocite{*} 
\bibliography{exemplo}

\end{document}